\documentclass{article}
\usepackage[utf8]{inputenc}
\usepackage[english]{babel}
\usepackage{amssymb,array}
\usepackage{enumitem}
\usepackage[usenames, dvipsnames]{color}
\usepackage[margin=0.5in]{geometry}
\usepackage{amsmath}
\usepackage{cancel}

\begin{document}

\null\hfill\begin{tabular}[t]{l@{}}
  \textbf{HW. 5 Ch121 Atomic Simulations}\\
  \textit{Winter 2017}\\
  \textit{Email to: Daniel Brooks (daniel.brooks@caltech.edu)}
\end{tabular}
\vspace{10mm}

\noindent

\section*{Purpose Of This Homework}
\noindent\rule[0.5ex]{\linewidth}{1pt}\\

In this homework assignment we are going to become familiar with a common open source MD code known as LAMMPS (Large-scale Atomic/Molecular Massively Parallel Simulator). 
We will use LAMMPS to run molecular dynamics on a box of water which will give us a trajectory. 
Next we will utilize a code written in the Goddard group known as 2PT (Two Phase Thermodynamics) to calculate the thermodynamic properties of our system, from the trajectory. \\

After completing the calculation, you will prepare a brief report.\\

\section*{Materials To Be Reviewed Before The Lab Session}
\noindent\rule[0.5ex]{\linewidth}{1pt}\\

\begin{enumerate}
    \item 2PT User Manual
    \item Module Handout
    \item Briefly Review the original 2pt publication 'The Two Phase Model For Calculating Thermodynamic Properties ...' by S-T. Lin, M. Blanco, W. Goddard (just read the basic introduction to get an understanding of the problem do not worry about the theory).
\end{enumerate}

\section{Building Our Water Box}
\noindent\rule[0.5ex]{\linewidth}{1pt}\\

To begin we need a box of water that we can run molecular dynamics on. 
\begin{enumerate}

    \item Open VMD (on your personal computer) and open the TK console (located in Extensions). \\
   
    \item Once the console loads type the following\\
\% package require solvate \\

\item For our calculation lets study a system containing about 500 water atoms.
To do this lets assume a density of 1g/cm$^3$ and type the following into the TK consoule.\\
\% solvate -minmax \{ \{ 0 0 0 \} \{ a a a \} \}.\\
Here a should be the side length of your box which will give about 500 atoms (do not forget the two sets of brackets, and the spaces between the 0's and a's ). \\

\item Now you should see a box of water in your VMD window please record the actual number of atoms, and the number of water molecules within your system.\\

\item Now we need to save our file to send to the cluster.
In the VMD main window (that has the file, molecule, graphics..tab) click on your water box and then right-click your system and select save coordinates. 
Set the selected atoms to all and choose bgf as the file type, therefore the file should be called waterbox.bgf \\

\item Now that we have constructed a system we need to send it to the cluster to perform our calculations. 
Please use whatever software you want to send this file to your account on atom (mobaxterm, winscp, filezilla ....). 
As always make a new directory for this homework assignment. \\

\item Inside your 2pt homework directory make a directory called structure and place your bgf file there.
When VMD prepares the bgf format it makes some atom typing mistakes that we need to fix for running a LAMMPS calculation (if you try submitting the original file you will get an error). 
To fix these run the script in /ul/sflynn/scripts directory called fixbgf.pl, the path to this script is /net/hulk/home3/sflynn/scripts/fixbgf.pl
Therefore you could type \\
\% /ul/sflynn/scripts/fixbgf.pl waterbox.bgf
\\
Notice you should now have 2 files, the file with the Tilda is the original input file, and the corrected input file will now be called waterbox.bgf\\

\item We are now going to make the input file for our LAMMPS calculation. 
Make a new directory (at the same level as the structure) called lammps.
Copy the corrected waterbox.bgf structure to this lammps directory.
Running Lammps is a very common task in the group, therefore a script already exists to make the proper input files, written by Tod Pascal (/ul/tpascal).
Type the following command into the terminal to make the lammps input file.
This should all be one command, use tab complete to help you in typing this out and then press enter. \\

\% /ul/tpascal/scripts/createLammpsInput.pl -b waterbox.bgf -s waterbox -f "/ul/tpascal/ff/WAT/spcew.ff"\\

As the script runs it will will write some notes to terminal about what it's doing, there should be something like 7 steps that complete. 
This script should give you multiple files: data.waterbox, in.waterbox\_singlepoint, waterbox.lammps.pbs, in.waterbox.\\

\item Now that we have a LAMMPS file we are going to make some changes to set up our specific trajectory calculation (please feel free to ask what these specific changes do).
For all of these changes be sure to keep the same spacing format as the remainder of the in.waterbox file. 
Open the in.waterbox file and type\\
\% log \hspace{10mm}  \$\{sname\}\_npt.log\\
This should go just before the \\
print \\
print ======================\\
print "NPT dynamics with an isotropic pressure of 1 atm"\\
print =====================\\
print\\

\item Inside the NPT dynamics section make the following changes as well:\\
dump \hspace{10mm} 1 all custom \cancel{5000} 500 \${sname}.npt.lammps id type xu yu zu vx vy vz\\
run     \hspace{10mm}    \cancel{7500000} 50000 \# run for \cancel{15} 0.1 ns\\

\item Finally at the end of the file add the following lines:\\
kspace\_style    ewald 1e-04\\
compute     \hspace{10mm}    atomPE all pe/atom\\
compute     \hspace{10mm}    atomKE all ke/atom\\
variable    \hspace{10mm}    atomEng atom c\_atomPE+c\_atomKE\\
dump        \hspace{12mm}    4 all custom 1 \$\{sname\}\_2pt.atom.eng id v\_atomEng\\
run         \hspace{15mm}    0\\
undump      \hspace{8mm}    4\\
uncompute   \hspace{6mm}    atomPE\\
uncompute   \hspace{6mm}    atomKE\\

\item We have now completed our input lammps file to run a molecular dynamics calculation on this waterbox.
All that is left is  to submit this calculation to the cluster. 
To do this lets make another directory called lammp\_pbs.
Inside this directory copy over the data.waterbox file, and the in.waterbox files to this new directory. 
The other files (waterbox.bgf, in.waterbox\_singlepoint, and waterbox.lammps.pbs) are not needed for the calculation we are running and can be removed if you want. \\

\item In the lammps\_pbs directory make a new file called waterbox.pbs

In this file write the following lines:\\
\#!/bin/csh\\
 \#PBS -l walltime=500:00:00\\
 \#PBS -l nodes=1:ppn=2\\
 \#PBS -l pmem=4GB\\
 \#PBS -N waterbox\\
 \#PBS -q workq\\
 cd \$PBS\_O\_WORKDIR\\
 module load use.own\\
 module load saladimodules\\
 module load openmpi-x86\_64\\
 module load lammps\\
 mpirun -np 2 lmp -in in.waterbox\\

 Your lammps\_pbs directory should now  have a data.waterbox, in.waterbox, and waterbox.pbs.

 At this time if you have not loaded the modules Shyam maintains, please go do this (the calculation will not work otherwise). 
 To be clear all you need to do is return to your main directory (type cd).
 Then load use.own, and the private modules, once you do these two things you are all set!
 
 Submit your calculation and wait for the results.

The first line of the submission script tells the computer what language should be used, our cluster uses a csh environment (.cshrc).
Line 2 defines how long the job should be run for, it is good form to give a wall time that is reasonably close to how long your job should take, however, we will ignore good form.
Line 3 defines how many compute nodes we want to use for this calculation (in this case 1 node) and ppn is the number of processors (here we use 2). 
These lines can be changed depending on the work you are doing, lammps allows for parallel computing (using multiple nodes/processors) which can greatly decrease the time it takes for a calculation to run. 
Line 4 sets the memory to be allocated to this job.
Line 5 sets the name for the job, this name will appear when you use qstat after submitting the job to the cluster. 
Line 6 specifies which queue to submit the job to (our cluster only has one Q so this is not necessary for our work).
Line 7 Sets the directory to your current directory (therefore the calculation will output all of its results to the directory where you wrote the qsub file). 
The four module lines specify the lammps module that Shyam Saladi has created. 
Lammps is an opensource code meaning it changes periodically, a module allows one person to maintain the software and everyone else just chooses to run whatever software the module has.
This allows us to avoid worrying about having each person compile their own lammps. 
Finally the last line says to use mpirun, a software that allows lammps to use parallel computing, it then says how many processors mpi should use, it calls on the lamps package (lmp) and then specifies the lammps "in file". 
\end{enumerate}

\section{2PT Calculations On Water}
\noindent\rule[0.5ex]{\linewidth}{1pt}\\

Now that we have a lammps trajectory we may want to analyze it. 
There are various ways and methods to do this, depending on what you are trying to understand. 
For this assignment we are going to use a code and methodology written in the group known as 2PT. 
\begin{enumerate}
    \item Inside of your lammps\_pbs directory make a new directory called 2pt, and then copy over your data.waterbox file, and the waterbox.2pt.lammps trajectory.\\
2pt is a stand alone code that can be used to derive thermodynamic data from a molecular dynamics trajectory (which we calculated in LAMMPS). 
The software was written in the Goddard group and used extensively by /ul/tpascal.
We will use the code Tod has compiled on his account to run our calculation.\\
To run our 2pt calculation we will need an in file for 2pt (waterbox\_2pt.in), a lammps trajectory (waterbox.2pt.lammps), the lammps data file (data.waterbox), and a submission file (2pt.pbs).

\item We need to make an input file for the analysis, make a file called waterbox\_2pt.in\\
Inside of the file add the following lines. \\
IN\_LMPDATA data.waterbox\\
IN\_LMPTRJ waterbox.2pt\\
ANALYSIS\_FRAME\_INITIAL 1\\
ANALYSIS\_FRAME\_FINAL 0\\
ANALYSIS\_FRAME\_STEP 1\\
ANALYSIS\_VAC\_CORLENGTH 0.5\\
ANALYSIS\_VAC\_MEMORYMB 5000\\
ANALYSIS\_VAC\_2PT 3\\
ANALYSIS\_OUT waterbox\_2pt\\
ANALYSIS\_LMP\_TSTEP 0.001\\
ANALYSIS\_VAC\_LINEAR\_MOL 0\\
ANALYSIS\_VAC\_ROTN\_SYMMETRY 2\\
ANALYSIS\_VAC\_FIXED\_DF N\\

Save the file.
Here the last line N is the number of atoms in your system (look at the beginning of your data file). 
Please look at the User Manual provided by Tod to see what commands are available for this software. \\

\item Now make your submission file, call it 2pt.pbs.
Write the following lines inside this file.\\
\#! /bin/csh\\
\#PBS -l nodes=1:ppn=1,walltime=72:00:00\\
\#PBS -q workq\\
\#PBS -N wbox\_2pt\\
cd \$PBS\_O\_WORKDIR\\
/ul/tpascal/codes/bin/md\_analysis waterbox\_2pt.in\\

\item We are ready to submit this thermodynamics analysis, submit your job to the cluster (it should take a few minutes to calculate).
\end{enumerate}

\section{2PT Analysis Of Water}
\noindent\rule[0.5ex]{\linewidth}{1pt}\\

After running this calculation please take a look at each file that is generated (use the manual from Tod to help you understand what is happening). 
Specifically look at the thermodynamics file and make sure you understand everything that we can learn about the system from doing a 2pt calculation.\\

What can we learn about the thermodynamics of water from this 2PT analysis? Summarize your results in a short report (1-2 pages). You are encouraged to discuss with other members of the class.\\

\section{Second half of the course}
\noindent\rule[0.5ex]{\linewidth}{1pt}\\

This is the final homework assignment for the class, the remainder of the course will be spent working on your individual research projects.
Therefore it is crucial that you have identified a project you would like to work on, and get in contact with people in the group who will be able to assist you with the project. 
Remember Bill is a key resource for all of this, he will be able to provide advice, and to help put you in contact with people in the group.
If for some reason you are unable to get in touch with group members Bill suggests, send Bill and email about it!\\

The final for the class will be a formal write-up of your original research. 

This report should consist of:
\begin{enumerate}
    \item Abstract 
    \item Introduction and Methods
    \item Results and Conclusions
\end{enumerate}
Below are some general comments meant to be suggestions, please make the report your own!

\subsection*{Abstract}
A brief review of the problem, the system, and the methods of analysis. Then provide key results. Make it short and concise.

\subsection*{Introduction and Methods}
Talk about the system we are studying, why it may be an important system, and why these types of calculations would be useful. 

\subsection*{Results and Conclusions}
Make sure your system and model are well defined for the reader.
Interpret your calculations and give some analysis of why the results are such. 
\end{document}
